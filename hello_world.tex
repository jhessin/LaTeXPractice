\documentclass[a4paper]{slides}

\title{My First Document}
\date{2019/11/10}
\author{Jim Hessin}

\usepackage{amsmath}

\begin{document}
% Example 1
\ldots when Einstein introduced his formula
\begin{equation}
e = m \cdot c^2 \; ,
\end{equation}
which is at the same time the most widely known
and the least well understood physical formula.
% Example 2
\ldots from which follows Kirchhoff's current law:
\begin{equation}
\sum_{k=1}^{n} I_k = 0 \; .
\end{equation}
Kirchhoff's voltage law can be derived \ldots

% Example 3
\ldots which has several advantages.
\begin{equation}
I_D = I_F - I_R
\end{equation}

is the core of a very different transistor model.\\
\ldots
%   \pagenumbering{gobble}
%   \maketitle
%   \newpage
%   \pagenumbering{arabic}
%
%   \section{Introduction}
%
%   Hello World!
%
%   \subsection{Subsection}
%
%   Structuring a document is easy!
%
%   \subsubsection{An even smaller section}
%
%   More text.
%
%   \paragraph{}
%
%   Some paragraph text
%
%   \subparagraph{}
%
%   Some subparagraph text.
%
%   \section{Math Equations}
%
%   Here is a formula block
%   using output Equation:
%   \begin{equation*}
%     f(x) = x^2
%   \end{equation*}
%   This formula $f(x) = x^2$ is an example of an inline equation. \\
%   And another block using output align
%   \begin{align*}
%     1 + 2 &= 3 \\
%     1 &= 3 - 2
%   \end{align*}
%
%   \section{Fractions and more}
%
%   \begin{align*}
%     f(x) &= x^2 \\
%     g(x) &= \frac{1}{x}\\
%     F(x) &= \int^a_b \frac{1}{3}x^3
%   \end{align*}
%   \section{Matrices}
%
%   Here is an example of building a matrix.
%   \begin{equation}
%     \left[
%   \begin{matrix}
%     1 & 0\\
%     0 & 1
%   \end{matrix}
%     \right]
% \end{equation}
% \section{Brackets and Parenthesis}
% Both brackets and parethesis can be scaled in math mode like this:
% \begin{equation}
%   \left(\frac{1}{\sqrt{x}}\right)
% \end{equation}
\end{document}
