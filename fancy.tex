\documentclass[]{book}

% PREAMBLE
\usepackage[utf8]{inputenc}
\usepackage{fancyhdr}
% \usepackage[hmarginratio=1:1]{geometry}

\pagestyle{fancy}

% In the following this is the meaning of these characters

% E - Even pages
% O - Odd pages

% L - Left field
% R - Right field
% C - center field

% H - Header
% F - Footer

%\fancyhf{} %clear head and footer fields
\fancyhead{} % clear all fields
\fancyhead[LO,RE]{\thepage}
% \lhead{This is Left side of my header}
\chead{Hessin}
% \rhead{Right of head}

\fancyfoot{} % clear all fields
% \lfoot{Left of footer}
\cfoot{Hessin}
% \rfoot{Right field of footer}
\renewcommand{\footrulewidth}{0.5pt}

%\fancyhead[CO]{ODD}
%\fancyhead[CE]{Even}

%\rfoot{\thepage}
%\renewcommand{\headrulewidth}{2.5pt}

\title{Book about Fancy Headers and Footers}
\author{Jim Hessin}
\date{December  1, 2019}

% -- END PREAMBLE

\begin{document}
\maketitle
In this lesson we'll learn how to make various types of headers and footers using "fancyhdr" package. Since we also want to practice different header and footer on odd and even pages, we are using book class.  
 
 \chapter{Fancy Headers and Footers}

You may like to use fancy headers and footer for any of the following reasons:

\begin{itemize}
\item When you want decorative lines
\item When you want three-part headers and footers instead of a single
\item When you want wider header and footer text which goes beyond the text limits of the document.
\item when you want separate header/footer for even/odd pages
\item when you want different header/footer for chapter pages
\end{itemize}
\pagebreak
There is some text after putting a page break on the second page of the chapter to show you how the class puts its own headers and footers. So, if you want to know what header or sometimes a footer, a class will put by deafult, what you can do is just remove or comment the commands which you are using from "fancyhdr" package and you will see the default header and footer options of the class which you are using. For, example book class puts a header by default in which only right field of the header is set. You can add left and centre field of your own but if you add the right field, the default option will be gone and your option will stay. The book class puts the section and chapter names on the page header by-default. So it depends how you want to use this.
\chapter{Design Your own H \& F}
The second chapter will show you some more hints on using fancy headers and footers. But before you can design your own headers and footers, you may notice that if you are using book or report classes, the definition of class also designing some headers for you. You can see the chapters or sections in your header even if you don't do anything. But you can override this by your own design.
\section{Three part Headers and Footers}
As you know fancy headers and footers have three parts. The left, center and right parts. When you use commands provided by "fancyhdr" package, you use something like the following:\\
\subsection{Designing Your Header}
$\backslash$lhead\{Write here whatever you want to put on left part of your header\}\\
$\backslash$chead\{Write here whatever you want to put in center part of your header\}\\
$\backslash$rhead\{Write here whatever you want to put on right part of your header\}\\
\subsection{Designing Your Footer}
$\backslash$lfoot\{Write here whatever you want to put on left part of your footer\}\\
$\backslash$cfoot\{Write here whatever you want to put in center part of your footer\}\\
$\backslash$rfoot\{Write here whatever you want to put on right part of your footer\}\\

\chapter{Design Separate H \& F on Odd-Even Pages}


You first should know the following 7 characters and what they mean in "fancyhdr" package:\\
E stands for even page\\
O stands for odd page\\
L for left field \\
R for right field\\
C for center filed\\
H for header\\
F for footer\\
\section{How to use Commands and Options}
Following are few examples which will help you understand and design your headers for even and odd pages:\\\\
$\backslash$fancyhead\{\}: Means clear all header fields\\\\
$\backslash$fancyfoot\{\}: Means clear all footer fields\\\\
$\backslash$fancyhf\{\}: Means clear all header AND footer fields\\\\
$\backslash$fancyhead$[RO,LE]$\{Write here what you want for \textbf{R}ight header field of an \textbf{O}dd page and \textbf{L}eft header field of an \textbf{E}ven page\}\\\\
$\backslash$fancyfoot$[LO,CE]$\{Write here what you want for \textbf{L}eft footer field of an \textbf{O}dd page and \textbf{C}enter footer field of an \textbf{E}ven page\}\\\\
By using the above 7 characters, you can make commands as per your requirement. 
How to make money from your videos (during live streaming and afterwards):
\section{Dummy Text Which Does not look Bad}
Making videos cost money and it would be a good idea to monetize your live streaming, and then the videos and get some cash.
Monetization during live video streaming:
First, we talk about live streaming, and if you are using Youtube live, then you can make some instant cash during your live stream! There is a feature called super chat which is recently introduced by YouTube.  You can activate this feature, and your fans would be able to chat with you during the streaming. Their chat would be highlighted during the video stream. This way you would be able to see their chat and their chance of getting a reply would likely to increase. However,  to be able to send this kind of chat message, they would have to pay.  You then get your share from youtube. It is probably the easiest way to monetize your live streaming. Super chat is, however, available to people having a certain number of subscribers and in few countries only.
Monetization of Prime Video Contents:
Prime video contents are not those which are of high quality but also if you are presenting them to the right audience.  If you have created such high-quality videos, then you have many options to make money out of them.  There is a big market for videos on demand. Traditionally, it was a challenge to distribute your film. You would have to go through the distribution channels. Now some companies still follow the same practice, but on some platforms, you can reach directly. There are three types of market places where you can sell your videos. Firest type are transactional platforms, where people pay a small money to see the video, the second category is subscription platforms (iTunes) where they pay a monthly or annual subscription to get access to videos and the third are based on advertisement (You Tube, Hulu, DailyMotion) where you get paid based on the ads played on your videos. A brief introduction to these market places is below:
How to make money from your videos (during live streaming and afterwards):
Making videos cost money and it would be a good idea to monetize your live streaming, and then the videos and get some cash.
Monetization during live video streaming:
First, we talk about live streaming, and if you are using Youtube live, then you can make some instant cash during your live stream! There is a feature called super chat which is recently introduced by YouTube.  You can activate this feature, and your fans would be able to chat with you during the streaming. Their chat would be highlighted during the video stream. This way you would be able to see their chat and their chance of getting a reply would likely to increase. However,  to be able to send this kind of chat message, they would have to pay.  You then get your share from youtube. It is probably the easiest way to monetize your live streaming. Super chat is, however, available to people having a certain number of subscribers and in few countries only.
Monetization of Prime Video Contents:
Prime video contents are not those which are of high quality but also if you are presenting them to the right audience.  If you have created such high-quality videos, then you have many options to make money out of them.  There is a big market for videos on demand. Traditionally, it was a challenge to distribute your film. You would have to go through the distribution channels. Now some companies still follow the same practice, but on some platforms, you can reach directly. There are three types of market places where you can sell your videos. Firest type are transactional platforms, where people pay a small money to see the video, the second category is subscription platforms (iTunes) where they pay a monthly or annual subscription to get access to videos and the third are based on advertisement (You Tube, Hulu, DailyMotion) where you get paid based on the ads played on your videos. A brief introduction to these market places is below:
How to make money from your videos (during live streaming and afterwards):
Making videos cost money and it would be a good idea to monetize your live streaming, and then the videos and get some cash.
Monetization during live video streaming:
First, we talk about live streaming, and if you are using Youtube live, then you can make some instant cash during your live stream! There is a feature called super chat which is recently introduced by YouTube.  You can activate this feature, and your fans would be able to chat with you during the streaming. Their chat would be highlighted during the video stream. This way you would be able to see their chat and their chance of getting a reply would likely to increase. However,  to be able to send this kind of chat message, they would have to pay.  You then get your share from youtube. It is probably the easiest way to monetize your live streaming. Super chat is, however, available to people having a certain number of subscribers and in few countries only.
Monetization of Prime Video Contents:
Prime video contents are not those which are of high quality but also if you are presenting them to the right audience.  If you have created such high-quality videos, then you have many options to make money out of them.  There is a big market for videos on demand. Traditionally, it was a challenge to distribute your film. You would have to go through the distribution channels. Now some companies still follow the same practice, but on some platforms, you can reach directly. There are three types of market places where you can sell your videos. Firest type are transactional platforms, where people pay a small money to see the video, the second category is subscription platforms (iTunes) where they pay a monthly or annual subscription to get access to videos and the third are based on advertisement (You Tube, Hulu, DailyMotion) where you get paid based on the ads played on your videos. A brief introduction to these market places is below:

\end{document}




