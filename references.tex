\documentclass{article}
%\usepackage{cite}

\begin{document}
\section{Introduction and Related Work}
Representation of the environment of a mobile robot is of
fundamental importance. If there is no prior information about the
environment the the robot has to build this during the exploration.
Over the past decade there has been a lot of work on map building
for mobile robots. Some approaches use the landmarks (installed or
natural) as a reference for map building while some use the relative
positions of landmarks as invariant features for map building.  The
major problem with all map building approaches is the data
association problem also known as correspondence problem. This
problem arises due to the fact that landmarks can not be identified
uniquely. There always exist an uncertainty associated with it. Scan
matching approaches are very popular for the purpose of map building
and localization because of the efficiency. By merging laser scans
at different locations, a more complete environment presentation can
be obtained. Most of the scan matching methods are based on
Iterative Closest Point  (ICP) algorithm \cite{Shewchuck},
and its variants, the idea borrowed from vision community.
  In the  field of robotics Lu and Milios described
 a method of data association in their highly influential paper. They were
the first to formulate the SLAM problem as a set of links between
robot poses and to formulate a global optimization algorithm. A
similar approach has been used by
 for the localization of a  mail delivery robot. Another
scan matching approach is presented by Biber\cite{minguez-metricbased} which is
independent of the correspondence problem. Gutman
used a combine scan matcher in AMOS project. Another
 method  use the exhaustive search strategy for corresponding points
 which can be used  as an initial alignment. Some other hybrid
approaches for environment modelling are Pradalier
Tomatis, Ortin, and Dufourd
. Hough transform based approach of scan matching
 are computationally very expensive. Some approaches
 use Expectation Maximization (EM) Algorithm
 to solve the data association problems in a hill
climbing manner but these can not be implemented in an on-line
fashion because EM algorithm requires multiple passes. The problem
with iterative methods used in most of the scan matching methods is
that those can not be used recursively in an efficient manner. This
is the reason due to which scan matching approaches are sometimes
considered incapable of handling large map building problems. We
present a method which reduces the computational burden and suitable
for real-time applications. Instead of finding a point to point
correspondence like ICP, we extract straight lines from the laser
scans and find the rotational and translational shifts for a proper
match. This shift is then used to update the pose reported by the
dead-reckoning. This reduces the computations by nearly a factor of
ten. The beauty of our method is that it aligns the scan locally in
a single iteration. For map building however, the global
localization is achieved by formulating the geometrical constraints
and solving those using conjugate gradient method.

\bibliographystyle{ieeetr}
\bibliography{MyReferences}
\end{document} 
