\documentclass[letterpaper,12pt]{article}

% PREAMBLE
\usepackage[utf8]{inputenc}
\usepackage[inline]{enumitem}

\title{Lists in Latex}
\author{Jim Hessin}
\date{December  1, 2019}
% -- END PREAMBLE

\begin{document}
\maketitle

\section{Lists in Latex}
There are three environments which can be used to show lists

\subsection{The itemize Environment}

A list with set left margin
\begin{itemize}[leftmargin=3in]
  \item The first item in the list
  \item The second item in the list
  \item The third item in the list
\end{itemize}
A list with a \ast{} marker

\begin{itemize}[label=$\ast$]
  \item The first item in the list
  \item The second item in the list
  \item The third item in the list
\end{itemize}
A list with a \star{} marker

\begin{itemize}[label=$\star$]
  \item The first item in the list
  \item The second item in the list
  \item The third item in the list
\end{itemize}

\subsection{The enumerate Environment}

Standard
\begin{enumerate}[]
  \item The first item
  \item The second item
  \item The third item
\end{enumerate}

Alphanumeric numbering
\begin{enumerate}[label=\alph*)]
  \item The first item
  \item The second item
  \item The third item
\end{enumerate}

roman numerals
\begin{enumerate}[label=\roman*)]
  \item The first item
  \item The second item
  \item The third item
\end{enumerate}

Roman Numerals
\begin{enumerate}[label=\fbox{\Roman*}]
  \item The first item
  \item The second item
  \item The third item
\end{enumerate}

Lists can be split as well
\begin{enumerate}[resume*]
  \item The fourth item
  \item The fifth item
  \item The sixth item
\end{enumerate}

\subsubsection{The Horizontal List}
\begin{enumerate*}
  \item The first item
  \item The second item
  \item The third item
\end{enumerate*}

Different symbols in a list
\begin{enumerate}
  \item [$-$] The first item
  \item [$\ast$] The second item
  \item [$\star$] The last item
\end{enumerate}

\subsection{The Description Environment}
\begin{description}
  \item[First] The description of the first item
  \item[Second] The description of the second item
  \item[Last] The description of the last item
\end{description}
\end{document}
